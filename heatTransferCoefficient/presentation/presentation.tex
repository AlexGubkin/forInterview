\documentclass[10pt,xcolor={dvipsnames,table},aspectratio=169]{beamer}

%\usetheme{Boadilla}
%\usecolortheme{wolverine}
\usecolortheme{dolphin}
%\setbeamercovered{transparent}
%\setbeamercolor{block body}{bg=yellow}

\addtobeamertemplate{navigation symbols}{}{%
	\usebeamerfont{footline}%
	\usebeamercolor[fg]{footline}%
	\hspace{1em}%
	\insertframenumber/\inserttotalframenumber
}

\usepackage{fontspec}
\defaultfontfeatures{Renderer=Basic,Ligatures={TeX}}
%    \setmainfont{CMU Serif}
%    \setsansfont{CMU Sans Serif}
\setmainfont{DejaVu Serif}
\setsansfont{CMU Sans Serif}
%    \setmonofont{CMU Typewriter Text}
\usepackage{unicode-math}
%    \setmathfont{Latin Modern Math}
%    \setmathfont{Libertinus Math}
%    \setmathfont{TeX Gyre DejaVu Math}
\setmathfont{STIX Two Math}
\usepackage[english,russian]{babel}
\usepackage{graphicx}
\usepackage{wrapfig}
\usepackage{animate}
\usepackage{ragged2e}
\usepackage{color}
\usepackage{pgfplots}
\usepackage{subcaption}
\usepackage[font=small,labelfont=bf]{caption}
\usepackage{svg}

\usepackage{tikz}
%\usetikzlibrary{
	%	mindmap,
	%	arrows, % стрелки
	%	shapes.misc, % фигуры
	%	chains, % цепочки
	%	positioning, % позиционирование элементов
	%	scopes, % создание дополнительных веток
	%	shadows % тени
	%}

\graphicspath{{pic/}{figs/}}

\author[Губкин]{А.С.~Губкин}

\title[Расчёт коэффициента теплоотдачи прямого канала прямоугольного сечения]{Расчёт коэффициента теплоотдачи прямого канала прямоугольного сечения}

\date[Тюмень 2024]{Июль, 2024 г.}

\begin{document}
    \frame{\titlepage}

    %SLIDE #
    \begin{frame}{}

        %\transdissolve[duration=0.1]
        \justifying
        \normalsize

        \frametitle{Задача}

        Определить коэффициент теплоотдачи $ \alpha $ прямого канала длиной $ l~=~400$~мм прямоугольного сечения ширины $ w~=~15 $~мм и высоты $ h~=~8 $~мм с толщиной стенки $ \delta~=~1 $~мм при протекании через него диоксида углерода. Скорость и температура газа на входе в канал соответственно $ U_{inlet}~=~20 $~м/с, $ T_{inlet}~=~423.15 $~К. Давление на выходе из канала $ p_{outlet}~=~4 $~МПа. Канал обменевается энергией с внешней средой при температуре $ T_{ext}~=~723.15 $~К с коэффициентом теплоотдачи $ \alpha_{ext}~=~1000 $~Вт/м$^2$К.

    \end{frame}{}

    %SLIDE #
    \begin{frame}{}

        %\transdissolve[duration=0.1]
        \justifying
        \normalsize

        \frametitle{Фазовая диаграмма $CO_{2}$}

        \begin{minipage}[b]{0.49\linewidth}
            Из фазовой диаграммы диоксида углерода видно, что при данном режиме фазовые переходы отсутствуют. Течение будет происходить в газовой фазе. \\

            Параметры критической точки: $ p_{c}~=~7.3773 $~МПа, $ T_{c}~=~304.128 $~К.
        \end{minipage}
        \hfill
        \begin{minipage}{0.49\linewidth}
            \begin{figure}
                \centering
                \includesvg[scale=0.4]{Carbon_dioxide_pressure-temperature_phase_diagram}
            \end{figure}
        \end{minipage}

    \end{frame}{}

    %SLIDE #
    \begin{frame}{}

        %\transdissolve[duration=0.1]
        \justifying
        \normalsize

        \frametitle{Математическая модель}

        \[
            \begin{aligned}
                &\begin{aligned}
                    &\frac{\partial \rho}{\partial t} + \symbf{\nabla} \cdot \symbf{\rho u} = 0 \\
                    &\frac{\partial \rho \symbf{u}}{\partial t} + \symbf{\nabla} \cdot \rho \symbf{u} \otimes \symbf{u}
                    =
                    - \symbf{\nabla} p
                    + \symbf{\nabla} \cdot \symbf{\tau}\\
                    &\frac{\partial \rho E}{\partial t} + \symbf{\nabla} \cdot \rho \symbf{u} E + \symbf{\nabla} \cdot \symbf{u} p
                    =
                    - \symbf{\nabla} \cdot \symbf{q}
                    + \symbf{\nabla} \cdot \symbf{\tau \cdot \symbf{u}}
                \end{aligned} \quad \symbf{x} \in \Omega_{flow} \\
                &\begin{aligned}
                    &\symbf{u} = 0 \\
                    \symbf{\nabla} &p = 0
                \end{aligned} \quad \symbf{x} \in \partial \Omega_{wall} \\
                &\begin{aligned}
                    \symbf{\nabla} &\symbf{u} = 0 \\
                    &p = p_{1}
                \end{aligned} \quad \symbf{x} \in \partial \Omega_{in} \\
                &\begin{aligned}
                    \symbf{\nabla} &\symbf{u} = 0 \\
                    &p = p_{2}
                \end{aligned} \quad \symbf{x} \in \partial \Omega_{out}
            \end{aligned}
        \]

    \end{frame}{}
\end{document}